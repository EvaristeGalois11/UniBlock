\chapter{Concetti essenziali sulla blockchain}

Nella sua pi� primitiva definizione una blockchain � una struttura dati in grado di registrare informazioni e garantirne l'immutabilit� nel tempo ridondando i dati su un gran numero di nodi senza che sia necessaria un'entit� centrale che vigili su possibili minacce o malfunzionamenti. Concepita nel 2008 da Satoshi Nakamoto \cite{satoshi} come mezzo per decentralizzare il mondo delle transazioni finanziarie, � stata oggetto nel corso del tempo di un forte interesse accademico e commerciale venendo applicata e sviluppata in un gran numero di scenari diversi. Pur essendo presenti sul panorama attuale svariati esempi di blockchain anche molto diverse tra loro in complessit� e implementazione, possiamo individuare alcune primitive di base la cui analisi permette di categorizzare e partizionare il vasto insieme delle blockchain.

\section{Modalit� di accesso alla rete}
Una prima differenza fondamentale delle blockchain � la modalit� con cui i nuovi nodi possono entrare a far parte della rete. Nelle blockchain pubbliche qualunque dispositivo pu� diventare un nodo della rete senza alcun controllo sulla sua legittimit�. Per questo motivo tali blockchain prendono il nome di permissionless blockchain in quanto non c'� alcuna differenza gerarchica tra i vari nodi e chiunque pu� entrare a farne parte. In generale questo tipo di blockchain presenta un numero di partecipanti molto elevato grazie alla bassa soglia di entrata. Nelle blockchain private invece l'entrata nella rete � preceduta da una fase di autenticazione del soggetto come l'appartenenza a una determinata azienda. Tali blockchain sono quindi confinate nelle realt� che le sviluppano e le loro pool di nodi sono di conseguenza molto ridotte in quanto solo chi � qualificato pu� entrare a farne parte. Inoltre riflettendo la gerarchia dell'organizzazione proprietaria della blockchain sono in genere presenti differenze di responsabilit� tra i vari nodi, da qui il nome di permissioned blockchain. Possiamo identificare inoltre un terzo tipo di blockchain, quello ibrido, in cui vengono uniti alcune caratteristiche delle blockchain pubbliche con quelle private. Nelle blockchain ibride alcune funzioni sono lasciate come pubbliche mentre altre richiedono una preventiva autenticazione. Le blockchain pubbliche sono di gran lunga il tipo pi� comune di blockchain, ma quelle private stanno guadagnando terreno avendo attirato l'attenzione del mondo finanziario \cite{HELLIAR2020102136}.

\section{Algoritmo di consenso}
algoritmo di consenso

\section{Privacy e sicurezza}
privacy e sicurezza
bitcoint non criptato
etherum c'� il concetto di privacy

\section{Confronto di alcune blockchain}
riferimenti a bitcoin, ethereum, hyperledger (permissioned)

tabella di confronto
righe -> caratteristiche
colonne -> blockchain (compresa la mia)

Ci possono essere blockchain pubbliche e private.
