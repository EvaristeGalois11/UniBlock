\chapter{Concetti essenziali sulla blockchain}

Nella sua pi� primitiva definizione una blockchain � una struttura dati in grado di registrare informazioni e garantirne l'immutabilit� nel tempo ridondando i dati su un gran numero di nodi senza che sia necessaria un'entit� centrale che vigili su possibili minacce o malfunzionamenti. Concepita nel 2008 da Satoshi Nakamoto \cite{satoshi} come mezzo per decentralizzare il mondo delle transazioni finanziarie, � stata oggetto nel corso del tempo di un forte interesse accademico e commerciale venendo applicata e sviluppata in un gran numero di scenari diversi. Pur essendo presenti sul panorama attuale svariati esempi di blockchain anche molto diverse tra loro in complessit� e implementazione, possiamo individuare alcune primitive di base la cui analisi permette di categorizzare e partizionare il vasto insieme delle blockchain.

\section{Modalit� di accesso alla rete}
pubbliche e private\break
con permessi e senza permessi

\section{Algoritmo di consenso}
algoritmo di consenso

\section{Privacy e sicurezza}
privacy e sicurezza
bitcoint non criptato
etherum c'� il concetto di privacy

\section{Confronto di alcune blockchain}
riferimenti a bitcoin, ethereum, hyperledger (permissioned)

tabella di confronto
righe -> caratteristiche
colonne -> blockchain (compresa la mia)

Ci possono essere blockchain pubbliche e private.
