\chapter{Concetti essenziali sulla blockchain}
Nella sua pi� primitiva definizione una blockchain � una struttura dati in grado di registrare informazioni e garantirne l'immutabilit� nel tempo ridondando i dati su un gran numero di nodi senza che sia necessaria un'entit� centrale che vigili su possibili minacce o malfunzionamenti. Concepita nel 2008 da Satoshi Nakamoto \cite{satoshi} come mezzo per decentralizzare il mondo delle transazioni finanziarie, � stata oggetto nel corso del tempo di un forte interesse accademico e commerciale venendo applicata e sviluppata in un gran numero di scenari diversi. Pur essendo presenti sul panorama attuale svariati esempi di blockchain anche molto diverse tra loro in complessit� e implementazione, possiamo individuare alcune primitive di base la cui analisi permette di categorizzare e partizionare il vasto insieme delle blockchain.

\section{Modalit� di accesso alla rete}
Una prima differenza fondamentale delle blockchain � la modalit� con cui i nuovi nodi possono entrare a far parte della rete. Nelle blockchain pubbliche qualunque dispositivo pu� diventare un nodo della rete senza alcun controllo sulla sua legittimit�. Per questo motivo tali blockchain prendono il nome di permissionless blockchain in quanto non c'� alcuna differenza gerarchica tra i vari nodi e chiunque pu� entrare a farne parte. In generale questo tipo di blockchain presenta un numero di partecipanti molto elevato grazie alla bassa soglia di entrata. Nelle blockchain private invece l'entrata nella rete � preceduta da una fase di autenticazione del soggetto come l'appartenenza a una determinata azienda. Tali blockchain sono quindi confinate nelle realt� che le sviluppano e le loro pool di nodi sono di conseguenza molto ridotte in quanto solo chi � qualificato pu� entrare a farne parte. Inoltre riflettendo la gerarchia dell'organizzazione proprietaria della blockchain sono in genere presenti differenze di responsabilit� tra i vari nodi, da qui il nome di permissioned blockchain. Possiamo identificare inoltre un terzo tipo di blockchain, quello ibrido, in cui vengono uniti alcune caratteristiche delle blockchain pubbliche con quelle private. Nelle blockchain ibride alcune funzioni sono lasciate come pubbliche mentre altre richiedono una preventiva autenticazione. Le blockchain pubbliche sono di gran lunga il tipo pi� comune di blockchain, ma quelle private stanno guadagnando terreno avendo attirato l'attenzione del mondo finanziario \cite{HELLIAR2020102136}.

\section{Algoritmo di consenso}
Data la natura distribuita di una blockchain � di fondamentale importanza la ricerca del consenso tra i nodi, ovvero che ogni transazione generata venga validata e diffusa attraverso la rete rimanendo inalterata. Il primo algoritmo di consenso concepito � la proof of work: l'immutabilit� dei dati � garantita dalla difficolt� di computare rapidamente un puzzle crittografico. Nella rete Bitcoin ad esempio un nuovo blocco � considerato valido quando viene trovato un numero tale che inserito nell'header del blocco stesso rende il suo hash inferiore a una cert� quantit�. Variando tale quantit� � possibile modulare il carico dei blocchi generati dall'intera rete, dando il tempo ai blocchi di diffondersi tra i vari nodi. Ogni nodo d� la sua fiducia alla catena di blocchi in cui � stata spesa la maggior quantit� di tempo computazionale. Un algoritmo di consenso concepito pi� recentemente � la proof of stake: i nuovi blocchi vengono validati da nodi scelti casualmente in base a quanta valuta hanno investito nella rete. Pur ritenendo un certo livello di casualit�, l'algoritmo di selezione del prossimo validatore privilegia i maggiori scommettitori. La legittimit� della rete � garantita perci� dal fatto che chi ha investito maggirmente avr� interesse nel suo corretto funzionamento. La rete Ethereum prevede di effettuare il cambio da proof of work a proof of stake nei prossimi anni. Per un'analisi dettagliata sugli algoritmi di consenso fare riferimento a \cite{8632190}.

\section{Privacy e sicurezza}
privacy e sicurezza\newline
bitcoint non criptato\newline
etherum c'� il concetto di privacy\newline

\section{Confronto di alcune blockchain}
Di seguito una comparazione essenziale delle due pi� grandi blockchain moderne con UniBlock:
\newline
\newline
\centerline{
\begin{tabular}{ |c||c|c|c| } 
 \hline
& \large{\textit{\textbf{Bitcoin}}} & \large{\textit{\textbf{Ethereum}}} & \large{\textit{\textbf{UniBlock}}} \\ 
 \hline
 \hline
 \large{\textit{\textbf{Accesso}}} & Pubblica & Pubblica & Ibrida \\ 
 \hline
 \large{\textit{\textbf{Algoritmo di consenso}}} & Proof of work & Proof of work & Proof of work \\ 
 \hline
 \large{\textit{\textbf{Privacy}}} & Non crittografata & Non crittografata & Crittografata \\
 \hline
\end{tabular}
}
