\myChapter{Concetti essenziali di crittografia}

\section{Hash crittografico}
Una funzione hash � una funzione che trasforma un input di dimensione arbitraria in un output di dimensione fissa. Per poter essere impiegata in crittografia una funzione hash deve rispettare forti vincoli di sicurezza:
\begin{description}
  \item[Resistenza alla pre-immagine] Dato un particolare hash \(h\), dev'essere estremamente difficoltoso trovare un input che abbia come hash \(h\). La funzione di hash deve quindi risultare una funzione one-way, cio� una funzione la cui inversa � impossibile da costruire o computazionalmente non eseguibile in tempi accettabili.
  \item[Seconda resistenza alla pre-immagine] Dato un un particolare input \(a\), dev'essere estremamente difficoltoso trovare un altro input che abbia lo stesso hash di \(a\).
  \item[Resistenza alle collisioni] Dev'essere estremamente difficoltoso trovare due input che abbiano lo stesso hash.
\end{description}

\section{Crittografia simmetrica}
Gli algoritmi a chiave simmetrica impiegano la stessa chiave per cifrare un testo e per decifrarlo successivamente. Le due tipologie principali di algoritmi a chiave simmetrica sono:
\begin{description}
  \item[Cifrari a blocchi] L'algoritmo prende in input un numero fisso di bit e li cripta combinandoli con la chiave restituendo lo stesso numero di bit criptati in uscita. Gli unici input accettabili sono di dimensione pari al blocco di bit usato, qualunque altro input dovr� essere opportunamente allungato se troppo corto o troncato se troppo lungo. Solo usando l'algoritmo in particolari modali� di operazione potr� essere possibile permettere input di dimensione arbitraria.
  \item[Cifrari a flusso] L'algoritmo combina l'input con un flusso pseudorandom di bit derivato dalla chiave. Al contrario dei cifrari a blocchi non presentano problemi di dimensione dell'input, in quanto il flusso generato verr� calibrato esattamente della lunghezza necessaria per ciascun input.
\end{description}

\section{Modalit� di operazione}
Data l'estrema restrittiva dei cifrari a blocchi che permettono di criptare solo un blocco di bit di dimensione fissa, sono stati concepiti diversi modi di usare tali cifrari per garantirne certe propriet� o per renderne pi� agevole l'uso in alcune situazioni:
\begin{description}
  \item[ECB] L'input viene partizionato in blocchi di dimensione fissa e ognuno viene criptato singolarmente.
  \item[CBC] L'input viene partizionato in blocchi di dimensione fissa e criptato mettendolo in xor con il blocco criptato precedente.
  \item[CTR] L'input viene partizionato in blocchi di dimensione fissa e ogni blocco viene criptato con una chiave generata a partire da una funzione sequenziale. Permette di fatto di trasformare un cifrario a blocchi in uno a flusso.
  \item[GCM] Modalit� molto simile a CTR in quanto permette tramite la derivazione sequenziale di una chiave di trasformare un cifrario a blocchi in uno a flussi. Aggiunge mediante l'uso dei campi di Galois l'aggiunta di una firma che garantisce l'integrit� del processo di cifrazione.
\end{description}

\section{Crittografia asimmetrica}
Gli algoritmi a chiave pubblica impiegano due chiavi diverse, una per cifrare e una per decifrare. Quella per cifrare � chiamata chiave pubblica in quanto non costituisce segreto ed � liberamente distribuibile senza protezioni. Quella per decifrare invece � chiamata chiave segreta e deve essere mantenuta appunto segreta e nota solo alla persona che la possiede. Tra le due chiavi � presente una relazione matematica che solitamente implica una funzione one-way sottostante, in quanto deve essere molto facile generare una chiave pubblica a partire da una chiave privata ma il contrario non deve essere computazionalmente fattibile. Oltre che cifrare � possibile anche firmare un input con la propria chiave privata cos� da garantire la sua autenticit�. Chiunque per mezzo della chiave pubblica potr� verificare infatti che la firma proviene dalla chiave privata corrispondente.

\section{Crittografia ellittica}
La crittografia ellittica � una particolare famiglia di algoritmi a chiave pubblica che si basano sull'uso di curve ellittiche \(y^{2}=x^{3}+ax+b\) su campi finiti. La funzione one-way che protegge la relazione tra le due chiavi � quella del logaritmo discreto. Al contrario di altri algoritmi a chiave pubblica come RSA (basato sulla fattorizzazione in numeri primi) permettono di avere chiavi di dimensioni molto pi� contenute e un'efficienza di esecuzione molto elevata.

\section{Scambio di chiavi tramite Diffie-Hellman}
Lo scambio di chiavi tramite Diffie-Hellman permette a due soggetti di scambiarsi tramite un canale insicuro le proprie chiavi pubbliche, combinandole con le chiavi private in modo tale da giungere ognuno separatamente a una certa quantit�. Tale quantit� risolter� identica per entrambi grazie alle propriet� matematice che collegano le chiavi in gioco. Un possibile attaccante in ascolto sul canale non potr� calcolare la stessa quantit� in quanto le rispettive chiavi private non hanno mai transitato nel canale, risultando quindi ignote all'attaccante e impossibili da ottenere a partire da quelle pubbliche. La quantit� comune ottenuta � tipicamente utilizzata come base per derivare una chiave simmetrica.

\section{Algoritmi impiegati}
\begin{description}
   \item[SHA-3] Algoritmo di hashing crittografico pensato come futuro successore della famiglia di algoritmi SHA-2 \cite{sha3}. Usato in UniBlock come puzzle crittografico alla base della proof of work.
   \item[AES] Algoritmo di cifratura simmetrica a blocchi diventato lo standard de facto per questo genere di algoritmi avendo supporto a livello di assembly nella maggior parte delle CPU moderne \cite{aes}. La dimensione dei blocchi � pari a 128 bit. Usato in UniBlock in modalit� GCM per criptare il contenuto degli eventi.
   \item[X25519] Algoritmo di scambio di una chiave tramite Diffie-Hellman basato sulla curva ellittica Curve25519 \cite{x25519}. Utilizzato in UniBlock per negoziare tra due utenti una chiave comune.
   \item[Ed25519] Algoritmo di firma a chiave pubblica basato sulla curva ellittica Curve25519 \cite{ed25519}. Usato in UniBlock per firmare gli eventi generati da un utente.
\end{description}
