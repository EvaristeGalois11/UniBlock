\myChapter{Concetti essenziali di crittografia}
UniBlock impiega varie tecnologie di cifratura moderne. Una disamina dettagliata del processo completo sar� oggetto dei prossimi capitoli, daremo ora una panoramica generale dei principali algoritmi usati.

\begin{description}
   \item[SHA-3] Algoritmo di hashing crittografico pensato come futuro successore della famiglia di algoritmi SHA-2 \cite{sha3}. Usato in UniBlock come puzzle crittografico alla base della proof of work.
   \item[AES] Algoritmo di cifratura simmetrica a blocchi diventato lo standard de facto per questo genere di algoritmi avendo supporto a livello di assembly nella maggior parte delle CPU moderne \cite{aes}. Permette di criptare simmetricamente, ovvero per mezzo di un'unica chiave sia in cifratura che in decifratura, un blocco di 128 bit. Per criptare pi� di 128 bit vengono usate particolari modalit� di operazione come ECB, in cui l'inseme di bit viene partizionato in gruppi di 128 bit e criptati singolarmente. UniBlock usa la modalit� di operazione GCM, modalit� estremamente efficiente che garantisce oltre alla cifratura anche l'integrati� dei dati.
   \item[X25519] Algoritmo di scambio di una chiave tramite Diffie-Hellman basato sulla curva ellittica Curve25519 \cite{x25519}. L'agloritmo di Diffie-Hellman permette a due soggetti di scambiarsi tramite un canale insicuro delle informazioni pubbliche, ad esempio le rispettive chiavi pubbliche di un cifrario a chiave pubblica, e combinarle in modo tale da raggiungere ognuno distintamente a una chiave uguale per entrambi cos� da poterla usare ad esempio per inizializzare una cifratura simmetrica. Nell'algoritmo X25519 le chiavi pubbliche scambiate sono coordinate sul campo finito della curva ellittica Curve25519, che verranno combinate con le rispettive chiavi private per giungere a una coordinata comune.
   \item[Ed25519] Algoritmo di firma a chiave pubblica basato sulla curva ellittica Curve25519 \cite{ed25519}. Data una coppia di chiavi, una pubblica e una privata, l'algoritmo permette di generare una firma mediante la chiave privata cos� che chiunque possa verificare tramite la chiave pubblica che il messaggio non � stato alterato ed � stato prodotto dalla persona che possiede la chiave privata.
\end{description}
