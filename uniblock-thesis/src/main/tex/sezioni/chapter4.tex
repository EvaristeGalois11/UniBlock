\myChapter{Applicazione UniBlock}
Bench� la struttura descritta nel capitolo precedente sia agnostica nella sua applicazione, UniBlock � stata realizzata specificatamente per un ambiente universitario. I suoi utenti sono per tanto i vari componenti accademici che interagiscono tra di loro tramite la produzione di eventi.
\begin{description}
  \item[Professori] Utenti in possesso di un certificato di ruolo PROFESSOR. Possono generare eventi come la pubblicazione di una data di appello o l'esito di un esame sostenuto da uno studente.
  \item[Studenti] Utenti in possesso di un certificato di ruolo STUDENT. Possono generate eventi come l'iscrizione a un appello o l'accettazione di un esito di un esame.
  \item[Segreteria] L'entit� centrale che controlla i certificati dei vari utenti viene identificata dalla segreteria didattica e organi analoghi. Il suo compito infatti � quello di verificare i dati di un nuovo immatricolato e generare il nuovo certificato. Similmente alla nuova assunzione di un membro del corpo docenti a seguito dei debiti controlli generer� il certificato per il neo assunto.
\end{description}

\section{Analisi Demo}
Una demo del progetto � disponibile al seguente link \url{https://hub.docker.com/r/evaristegalois/uniblock}. E' necessario un sistema OCI runtime qualunque per poterla eseguire come ad esempio docker o podman.

