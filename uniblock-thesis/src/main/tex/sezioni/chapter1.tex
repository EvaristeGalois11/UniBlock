\myChapter{Introduzione}
Attualmente l'infrastruttura tipica di un'universit� � prettamente centralizzata, sia per quanto riguarda la parte burocratica sia quella informatizzata. Tutti i dati vengono salvati per tanto in appositi database, accessibili solo dagli organi interni autorizzati. Una struttura dati centralizzata tuttavia pu� essere causa di una paralisi completa del servizio a seguito della perdita di informazioni. Incidenti come incendi o cortocircuiti o attacchi malevoli come ransomware o DOS sono solo alcuni esempi di eventi che rappresentano una minaccia all'integrit� dei dati conservati. Sono necessari quindi costosi processi di backup multi livello per assicurare la continuit� di servizio a fronte di problematiche. UniBlock si propone come soluzione a tutto ci� decentralizzando la struttura informatizzata accademica, pur ritenendo la gestione burocratica a pochi organi centrali. La ridondanza delle informazioni offerta dalla tecnologia blockchain su cui UniBlock si basa risulta immune alla perdita di dati, in quanto ogni nodo della rete possiede una copia esatta di tutti gli eventi. Si dovrebbero verificare incidenti simultanei su ogni singolo dispositivo della rete o un attacco mirato a ogni nodo per portare offline il sistema, in quanto anche una sola copia della blockchain rimasta sarebbe sufficiente a ricreare l'intera rete.

UniBlock permetterebbe inoltre di trasformare qualunque dispositivo in un nodo della rete universitaria. I PC di segreteria o quelli di laboratorio potrebbero venire usati per mantenere in background il servizio di gestione della rete blockchain, sfruttando l'inattivit� di tali dispositivi per minare i nuovi blocchi. Questo permetterebbe di evitare l'acquisto e il mantenimento di grossi server, riguadagnando l'inutilizzo di macchine gi� operanti all'interno dell'universit�.

Di seguito � descritta una PoC di implementazione di tale progetto volta a considerare solo gli aspetti base necessari a realizzare UniBlock. Il codice sorgente � liberamente consultabile a \url{https://github.com/Errore418/UniBlock}.
