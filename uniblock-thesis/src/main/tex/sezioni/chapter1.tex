\myChapter{Introduzione}
La tecnologia delle blockchain ha visto un notevole sviluppo nel corso dell'ultimo decennio. Nonostante progetti come Bitcoin o Ethereum siano architetturalmente molto complessi e le loro codebase estremamente estese, circa mezzo milione di righe di codice ciascuna, alla loro base hanno un concetto relativamente semplice, cio� quello di blockchain. Da queste premesse nasce quindi il tentativo di scrivere da zero una blockchain rudimentale in Java. Per differenziarla da progetti molto pi� completi � stato deciso di aggiungerle uno strato di crittografia sovrastante per proteggere la riservatezza delle informazioni registrate all'interno. UniBlock infine nasce come timida applicazione del tutto in uno scenario in cui tale blockchain venga inserita in un ipotetico contesto universitario.
