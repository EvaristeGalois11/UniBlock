\myChapter{Struttura UniBlock}
I dati all'interno di UniBlock sono organizzati in blocchi, ciascuno con un proprio header e una lista di eventi. L'header di un blocco contiene a sua volta le informazioni necessarie per identificare e validare il blocco all'interno della blockchain stessa: l'hash del blocco precedente, la difficolt� richiesta per minare il blocco attuale, l'hash di radice, ora e data di generazione del blocco, un numero variabile (nonce). L'hash del blocco precedente permette di collegare in sequenza i vari blocchi formando una catena. Solo un blocco pu� dichiararsi successore di un altro, altrimenti la catena si spezzerebbe in due rami. Per evitare che ci� accada UniBlock impiega una versione semplificata della proof of work della rete Bitcoin utilizzando la difficolt�, espressa come un numero intero positivo, e il nonce, anch'esso un numero intero positivo. L'hash di radice, invece, viene impiegato come forma di ottimizzazione del calcolo e successiva verifica dell'hash di un blocco organizzando la lista di eventi in un albero di Merkle.

Gli eventi sono il corrispettivo delle transazioni in una rete Bitcoin, ma avendo un contenuto di informazione arbitraria incapsulano un avvenimento successo all'interno della blockchain piuttosto che un mero movimento di denaro. Vengono previsti sostanzialmente due tipi di eventi: i certificati e gli eventi criptati. I certificati riguardano la registrazione sulla blockchain di un nuovo utente autorizzato, mentre gli eventi criptati contengono dati cifrati leggibili solo dai destinatari riportati nell'evento stesso. I certificati sono gestiti e pubblicati da un'entit� centrale che rappresenta l'ente proprietario della blockchain. Solo tale entit� infatti pu� autorizzare l'ingresso di un nuovo utente nella rete. Gli eventi criptati invece possono essere pubblicati da chiunque sia in possesso di un certificato valido.

\section{Blocco}
Ogni blocco contiene nel proprio header il riferimento al suo predecessore cos� da renderne estremamente difficoltosa la sua modifica, dal momento che comporterebbe la riscrittura di ogni suo successore. Affinch� il blocco sia ritenuto valido il suo hash, calcolato come l'hash SHA-3 della serializzazione dell'oggetto rappresentante il suo header, deve rispettare i criteri di difficolt� imposti, cio� deve iniziare con tanti zeri quanti riportati nel numero della difficolt�.
\begin{lstlisting}
private boolean isMined(BlockHeader blockHeader) {
  return hashService.hash(blockHeader)
          .startsWith("0".repeat(blockHeader.getDifficulty()));
}
\end{lstlisting}
In caso in cui l'hash non risultasse valido, il nonce verrebbe incrementato fino a trovare un hash corretto. Maggiori sono gli zeri iniziali da ricercare, minore � il numero di hash totali disponibili e di conseguenza minore � la probabilit� di minare correttamente il blocco.

Il calcolo di un hash � quindi un procedimento dispendioso che comporta prima la serializzazione di un oggetto in un flusso di byte e poi il calcolo dell'hash vero e proprio. Per evitare di serializzare un blocco intero ogni volta, l'hash � calcolato solo sull'header di un blocco. Per comprendere anche la lista di eventi nel calcolo dell'hash viene fatto uso di un albero di Merkle.
\begin{lstlisting}
public String calculateRootHash(List<? extends Event> events) {
  return getRootHash(events.stream()
  	.map(hashService::hash).toList());
}

private String getRootHash(List<String> hashes) {
  if (hashes.size() == 1) {
    return hashes.get(0);
  } else {
    return getRootHash(Lists.partition(hashes, 2)
            .stream().map(this::reduceHash).toList());
  }
}

private String reduceHash(List<String> hashes) {
  var first = hashes.get(0);
  var second = hashes.size() == 2 ? hashes.get(1) : first;
  return hashService.hash(first + second);
}
\end{lstlisting}
La lista di eventi viene prima convertita in una lista dei rispettivi hash. Successivamente la lista di hash viene partizionata in una lista di coppie di hash. Tali coppie vengono ridotte a un hash singolo calcolando l'hash della loro concatenazione. Nel caso sia presente un hash spaiato, verr� calcolato l'hash della sua concatenazione con se stesso. Tale procedimento si ripete fino ad ottenere un singolo hash, detto appunto hash di radice dell'albero.

L'hash di radice inserito nell'header di un blocco permette di garantire l'integrit� degli eventi senza dover calcolare ogni volta il loro hash. Una modifica a un evento di un blocco comporterebbe infatti una modifica in cascata a tutti gli hash dell'albero di Merkle fino alla radice.

\section{Evento}
Un evento rappresenta qualsiasi avvenimento avvenuto all'interno di UniBlock. Tali avvenimenti sono categorizzati in due tipi: i certificati identificano l'ingresso nella rete di un nuovo utente autorizzato, mentre gli eventi criptati sono un contenitore standard che incapsulano l'evento vero e proprio aggiungendo uno strato di crittografia. Solo gli utenti autorizzati possono produrre eventi criptati, mentre solo ed esclusivamente un ente centrale pu� registrare un nuovo utente rilasciando un certificato.

\subsection{Certificato}
I certificati sono eventi che identificano un utente all'interno della rete. Ad ogni nuovo utente infatti viene assegnata una doppia coppia di chiavi pubbliche e private, per un totale di quattro chiavi. Una coppia permette di firmare gli eventi pubblicati, mentre l'altra permette di generare e consultare eventi criptati. Le due chiavi pubbliche sono salvate nella blockchain nel rispettivo evento di certificato dell'utente, mentre le due chiavi private sono salvate sui dispositivi personali dell'utente non potendo venire pubblicate sulla rete. La scelta di usare due coppie di chiavi separate invece di una sola ha due motivazioni: la prima � di tipo implementativo in quanto gli algoritmi di firma e cifratura sono diversi e perci� usano chiavi di tipo diverso; bench� sia matematicamente possibile convertire una chiave in un'altra, in Java risulta particolarmente arduo effettuare tale operazione. L'altra motivazione � per garantire una maggiore sicurezza in quanto con una singola chiave un attaccante che riesce a entrarne in possesso pu� sia firmare nuovi eventi al posto dell'utente attaccato che leggere tutti i precedenti blocchi generati. La separazione in due chiavi permette quindi una maggiore flessibilit� di gestione, potendo ad esempio rigenerare una sola delle due senza compromettere l'altra.

Il certificato di un utente riporta anche il suo ID con cui viene identificato all'interno della rete. Tale ID � generato come l'hash della concatenazione delle sue due chiavi pubbliche.
\begin{lstlisting}
private String calculateUserId(PublicKey signPbk, PublicKey dhPbk) {
  return hashService.hash(concatPbk(signPbk, dhPbk));
}

private byte[] concatPbk(PublicKey signPbk, PublicKey dhPbk) {
  byte[] signPbkEncoded = signPbk.getEncoded();
  byte[] dhPbkEncoded = dhPbk.getEncoded();
  return ByteBuffer.allocate(signPbkEncoded.length + dhPbkEncoded.length)
      .put(signPbkEncoded)
      .put(dhPbkEncoded)
      .array();
}
\end{lstlisting}

Il certificato prevede inoltre la possibilit� di assegnare un ruolo all'utente. I ruoli consentono di controllare finemente i permessi degli utenti, ad esempio permettendo solo ad alcuni ruoli di generare determinati tipi di eventi criptati.

Anche l'entit� centrale possiede un proprio certificato contenuto nel blocco di genesi, cio� il primo blocco della catena. Tale certificato � gestito in modo particolare in quanto assunto come verificato senza per� presentare una firma. Le due coppie di chiavi del certificato di genesi consentono quindi all'entit� centrale di produrre nuovi certificati, in quanto qualunque altro certificato non in possesso di una firma valida verr� rigettato dalla catena.

\subsection{Evento criptato}
Un evento criptato � un evento che fa da contenitore a informazioni riservate. Nonostante possa racchiudere informazioni di natura arbitraria la sua struttura � costante: un evento contiene l'ID dell'utente che lo ha prodotto, il tipo di dati contenuti al suo interno, una lista di destinatari autorizzati alla lettura, il payload criptato e una firma che ne prova l'autenticit�. E' possibile verificare quindi l'autenticit� di un evento criptato verificando la firma al suo interno grazie al certificato di chi l'ha prodotto. La lista di destinatari invece � riportata grazie ai loro rispettivi ID e per ognuno viene generata una specifica chiave con cui poter accedere al contenuto del payload.

Quando un utente vuole produrre un evento dal contenuto riservato pu� indicare un numero arbitrario di destinatari a cui permettere l'accesso alle informazioni contenute.
\begin{lstlisting}
public EncryptedEvent build(Encryptable event, String author, List<String> receivers) {
  var payloadKey = aesService.randomKey();
  var payload = aesService.encrypt(payloadKey, hashService.serialize(event), false);
  var sign =
      pkService.sign(payload.getBytes(StandardCharsets.UTF_8), keyManager.retrieveSignPk(author));
  var eventContainer = new EncryptedEvent(author, event.getType(), payload, sign);
  Stream.concat(receivers.stream(), Stream.of(author))
      .map(id -> encryptKey(id, payloadKey, author))
      .forEach(e -> eventContainer.addKey(e.getKey(), e.getValue()));
  return eventContainer;
}

private Map.Entry<String, String> encryptKey(String id, byte[] key, String author) {
  PublicKey pbk = blockchain.searchCertificate(id).dhPbk();
  PrivateKey dhPk = keyManager.retrieveDhPk(author);
  return Map.entry(id, pkService.encrypt(dhPk, pbk, key));
}
\end{lstlisting}
Viene generata un'unica chiave casuale \(payloadKey\) con cui cifrare tramite AES l'intero contenuto dell'evento. Per ogni destinatario viene effettuato uno scambio di Diffie-Hellman combinando la chiave pubblica del destinatario con la chiave privata dell'autore dell'evento, giungendo a una chiave segreta condivisa. Tramite tale chiave condivisa viene cifrata tramite AES la chiave \(payloadKey\) e inserita in una mappa insieme a tutte le altre chiavi dei destinatari. Viene inoltre applicata tramite la chiave personale dell'autore una firma all'evento.

Quando un destinatario autorizzato vuole accedere al contenuto dell'evento svolge a ritroso i passi dell'algoritmo. Effettua uno scambio di Diffie-Hellman combinando la propria chiave personale con quella pubblica dell'autore, giungendo alla stessa chiave segreta condivisa ottenuta dall'autore. Tramite tale chiave condivisa pu� ottenere \(payloadKey\), decifrando la propria chiave dalla mappa dei destinatari. Tramite \(payloadKey\) pu� infine decifrare e leggere il contenuto dell'evento. Inoltre tramite la chiave pubblica dell'autore pu� verificare la legittimit� dell'intero processo.

La presenza di una chiave per ogni destinatario permette di criptare una sola volta l'evento, ottimizzando la struttura del blocco evitando di dover duplicare il contenuto criptato per ogni destinatario.

