\myChapter{Conclusioni}
UniBlock dimostra come sia possibile conciliare la comodit� di una blockchain pubblica con i requisiti di riservatezza di un ambiente accademico. Una sua applicazione su larga scala potrebbe portare a dismettere le grandi infrastrutture centralizzate riciclando il tempo di inattivit� dei dispositivi gi� distribuiti all'interno dell'organizzazione.

Al momento ogni evento deve riportare la lista completa di destinatari, ci� potrebbe risultare particolarmente dispendioso nella situazione in cui un professore debba pubblicare un appello di esame per un elevato numero di studenti. Inoltre nel caso in cui uno studente si aggiunga al corso in data successiva alla pubblicazione dell'evento di appello, il professore si ritroverebbe costretto a ripubblicare l'evento aggiungendo ai destinatari il nuovo studente. Per risolvere queste criticit� potrebbe venire introdotto una categoria di utenti che funge da gruppo. Tali gruppi andrebbero pertanto a rappresentare i vari corsi di studi, iscritti a un esame, commissioni di laurea, dipendenti di una particolare sezione o pi� in generale un qualunque sottoinsieme di utenti della rete. Sarebbe sufficiente pertanto riportare come destinatario dell'evento il singolo gruppo, cos� che tutti gli utenti che ne appartengono possano leggerne il contenuto. Ci� permetterebbe di risparmiare spazio nei singoli eventi e di rendere fruibili gli eventi anche a utenti aggiunti successivamente.

Un'altra criticit� di UniBlock � l'estrema difficolt� a far fronte a un furto di chiavi private data la natura immutabile di una blockchain. Nel caso infatti in cui un utente malintenzionato venisse in possesso delle chiavi segrete di un utente della rete, avrebbe accesso a tutti gli eventi riguardanti quel particolare utente. Potrebbe inoltre pubblicare a sua volta eventi legittimante firmati a nome dell'utente attaccato. Un furto delle chiavi private dell'entit� centrale potrebbe voler dire la totale compromissione dell'intera rete. Una parziale soluzione in caso di compromissione del certificato potrebbe essere la pubblicazione immediata di un nuovo certificato per l'utente, con un meccanismo di invalidazione di quello precedente cos� che tutti i nodi siano informati di dover rigettare eventi firmati con il certificato compromesso. Ci� purtroppo non � sufficiente ad impedire all'attaccante di poter comunque accedere a tutte le informazioni dell'utente. Sarebbe necessario rimuovere dalla blockchain tutti gli eventi compromessi e rigenerarli con il nuovo certificato. Ci� tuttavia in un sistema immutabile e distribuito come quello di una blockchain risulta di difficile implementazione, dal momento che la riscrittura dell'intera catena di eventi � un'operazione che richiede un enorme sforzo computazionale.
